\documentclass{ctexart} % 使用ctexart文档类,自动处理中文

\usepackage{amsmath} % 支持数学公式

\begin{document}

\subsection{OPTICS 聚类算法}

OPTICS(Ordering Points To Identify the Clustering Structure)是一种基于密度的聚类算法,用于识别任意形状和密度变化的簇。OPTICS 算法通过计算每个数据点的可达距离(Reachability Distance)和核心距离(Core Distance),并按照可达距离对数据点排序,从而构建簇的层次结构。首先,算法计算每个数据点的核心距离,即它到其邻域内满足最小点数要求的最远距离。然后,计算数据点之间的可达距离,即到达某个点所需的最小核心距离。根据可达距离对数据点进行排序,生成簇的层次结构。

核心距离的计算公式为:
\[
\text{core\_dist}(p) = \min_{p' \in \text{Neighbors}(p)} \text{dist}(p, p')
\]
可达距离的计算公式为:
\[
\text{reachability\_dist}(o, p) = \max(\text{core\_dist}(p), \text{dist}(o, p))
\]

其中,\(\text{dist}(p, p')\) 是数据点 \(p\) 和 \(p'\) 之间的距离,\(\text{Neighbors}(p)\) 是点 \(p\) 的邻域。

OPTICS 算法的优点在于它能够处理任意形状和密度的簇,并且无需预先指定簇的数量。缺点是算法在处理大规模数据集时计算复杂度较高,特别是在高维空间中。OPTICS 算法的时间复杂度为 \(O(n \log n)\),其中 \(n\) 为样本数量。虽然与 DBSCAN 类似,但 OPTICS 能更好地处理具有密度变化的复杂数据集。

\end{document}
