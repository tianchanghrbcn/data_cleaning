\documentclass{article}
\usepackage{amsmath}  % 包含数学公式的包
\usepackage{xeCJK}    % 处理中文的包,适用于XeLaTeX
\setCJKmainfont{SimSun} % 设置中文字体,如宋体

\begin{document}
\subsection{Gaussian Mixture Model (GMM)}

Gaussian Mixture Model (GMM) 是一种基于概率模型的聚类算法,它假设数据由多个高斯分布的混合组成。与 K-Means 不同,GMM 通过概率来表示每个数据点属于不同簇的可能性,而不仅仅是将数据点硬性地分配给某个簇。

GMM 使用期望最大化 (Expectation-Maximization, EM) 算法来估计每个高斯分布的参数,包括均值、协方差矩阵和混合系数。EM 算法的主要步骤包括:

\begin{itemize}
    \item \textbf{期望步骤 (E-Step)}: 计算每个数据点属于每个高斯分布的后验概率(责任值)。
    \item \textbf{最大化步骤 (M-Step)}: 使用这些概率更新每个高斯分布的参数,包括均值、协方差矩阵和混合系数。
\end{itemize}

GMM 的目标是通过以下概率密度函数描述数据:

\[
p(x) = \sum_{k=1}^{K} \pi_k \mathcal{N}(x | \mu_k, \Sigma_k)
\]

其中:
\begin{itemize}
    \item \( K \) 是高斯分布的数量(即簇数)。
    \item \( \pi_k \) 是第 \( k \) 个高斯分布的权重,满足 \(\sum_{k=1}^{K} \pi_k = 1\)。
    \item \( \mathcal{N}(x | \mu_k, \Sigma_k) \) 是第 \( k \) 个高斯分布的概率密度函数,\(\mu_k\) 为均值,\(\Sigma_k\) 为协方差矩阵。
\end{itemize}

GMM 的优点在于其灵活性,可以处理不同形状和大小的簇,并能进行软聚类,即一个数据点可以部分属于多个簇。然而,GMM 的计算复杂度较高,尤其是在高维数据集上。此外,GMM 对初始参数的选择较为敏感,可能会陷入局部最优。

\end{document}
