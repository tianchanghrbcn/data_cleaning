\documentclass{ctexart} % 使用ctexart文档类,自动处理中文

\usepackage{amsmath} % 支持数学公式
\usepackage[margin=1in]{geometry} % 设置页边距为1英寸

\title{六种常见聚类算法的分析与总结} % 设置标题
\author{常添} % 设置作者名称(如需)
\date{08/12/2024} % 设置日期为当前日期

\begin{document}

\maketitle % 显示标题、作者和日期

\section{常见聚类算法}

\subsection{K-Means 聚类算法}

K-Means 是一种基于距离的聚类算法,通过迭代将数据集划分为 \(K\) 个簇。算法首先随机选择 \(K\) 个初始质心,然后将每个数据点分配到最近的质心,接着更新质心为簇内数据点的均值。这个过程重复进行,直到质心位置稳定或达到最大迭代次数为止。

K-Means 的目标是最小化簇内数据点到其质心的总距离平方和,具体表达为:

\[
J = \sum_{k=1}^{K} \sum_{i \in C_k} \| x_i - \mu_k \|^2
\]

其中,\(x_i\) 表示数据点,\(\mu_k\) 为第 \(k\) 个簇的质心,\(C_k\) 是簇 \(k\) 中的所有数据点。通过最小化 \(J\) 值,算法不断优化簇的划分。

K-Means 算法的优点在于简单易实现,且在处理大规模数据集时效率较高。然而,它需要预先指定簇的数量 \(K\),且对初始质心选择较为敏感,不同的初始选择可能导致不同的聚类结果。此外,K-Means 更适用于凸形簇,对噪声和异常值的处理能力较弱。

K-Means 的平均时间复杂度为 \(O(K \cdot n \cdot T)\),其中 \(K\) 为簇数,\(n\) 为样本数量,\(T\) 为迭代次数。最坏情况下,复杂度为 \(O(n^{(K+2)/p})\),其中 \(p\) 为特征数量。处理高维数据时,算法的计算开销可能会增加,尤其是在初始质心选择不当的情况下。

\subsection{Gaussian Mixture Model (GMM)}

Gaussian Mixture Model (GMM) 是一种基于概率模型的聚类算法,它假设数据由多个高斯分布的混合组成。与 K-Means 不同,GMM 通过概率来表示每个数据点属于不同簇的可能性,而不仅仅是将数据点硬性地分配给某个簇。

GMM 使用期望最大化 (Expectation-Maximization, EM) 算法来估计每个高斯分布的参数,包括均值、协方差矩阵和混合系数。EM 算法的主要步骤包括:

\begin{itemize}
    \item \textbf{期望步骤 (E-Step)}: 计算每个数据点属于每个高斯分布的后验概率(责任值)。
    \item \textbf{最大化步骤 (M-Step)}: 使用这些概率更新每个高斯分布的参数,包括均值、协方差矩阵和混合系数。
\end{itemize}

GMM 的目标是通过以下概率密度函数描述数据:

\[
p(x) = \sum_{k=1}^{K} \pi_k \mathcal{N}(x | \mu_k, \Sigma_k)
\]

其中:
\begin{itemize}
    \item \( K \) 是高斯分布的数量(即簇数)。
    \item \( \pi_k \) 是第 \( k \) 个高斯分布的权重,满足 \(\sum_{k=1}^{K} \pi_k = 1\)。
    \item \( \mathcal{N}(x | \mu_k, \Sigma_k) \) 是第 \( k \) 个高斯分布的概率密度函数,\(\mu_k\) 为均值,\(\Sigma_k\) 为协方差矩阵。
\end{itemize}

GMM 的优点在于其灵活性,可以处理不同形状和大小的簇,并能进行软聚类,即一个数据点可以部分属于多个簇。然而,GMM 的计算复杂度较高,尤其是在高维数据集上。此外,GMM 对初始参数的选择较为敏感,可能会陷入局部最优。

\subsection{Affinity Propagation (AP) 聚类算法}

Affinity Propagation (AP) 是一种基于消息传递的聚类算法,通过在数据点之间传递“责任”(Responsibility, \(r(i, k)\))和“可用性”(Availability, \(a(i, k)\))消息,自动确定簇的数量和中心点。算法首先计算数据点之间的相似度 \(s(i, k)\),然后初始化责任和可用性矩阵。

AP 的目标是通过不断更新以下公式,使得责任和可用性值达到平衡:

\[
r(i, k) = s(i, k) - \max_{k' \neq k} \{a(i, k') + s(i, k')\}
\]
\[
a(i, k) = \min\left(0, r(k, k) + \sum_{i' \notin \{i, k\}} \max(0, r(i', k))\right)
\]

其中,责任 \(r(i, k)\) 表示数据点 \(i\) 作为簇中心 \(k\) 的适合度,可用性 \(a(i, k)\) 表示数据点 \(i\) 选择 \(k\) 作为簇中心的适合度。最终具有较高“可用性”和“责任值”的点被选择为簇中心。

AP 的优点在于不需要预先指定簇的数量,且能够处理任意形状的簇。缺点在于对相似度度量和偏好值敏感,且计算复杂度较高。AP 算法的时间复杂度为 \(O(n^2 \cdot \log(n))\),其中 \(n\) 为样本数量。

\subsection{Hierarchical Clustering (HC) 聚类算法}

层次聚类(Hierarchical Clustering, HC)是一种用于分析数据集内嵌层次结构的聚类算法。HC 通过反复地将数据点进行分割或合并来形成一个树状的簇结构,称为树状图(Dendrogram)。HC 的两种主要方法是自底向上(凝聚法)和自顶向下(分裂法)。

\textbf{凝聚法}(Agglomerative Method):从每个数据点自身作为一个簇开始,逐步合并最近的簇,直到所有点都被合并到一个簇中。合并步骤通常基于以下几种距离度量:

\begin{itemize}
    \item \textit{最小距离法}(Single Linkage):两个簇之间的距离定义为它们之间最近点的距离:
    \[
    d_{\text{single}}(C_i, C_j) = \min_{x \in C_i, y \in C_j} \text{dist}(x, y)
    \]
    \item \textit{最大距离法}(Complete Linkage):两个簇之间的距离定义为它们之间最远点的距离:
    \[
    d_{\text{complete}}(C_i, C_j) = \max_{x \in C_i, y \in C_j} \text{dist}(x, y)
    \]
    \item \textit{平均距离法}(Average Linkage):两个簇之间的距离定义为它们之间所有点的平均距离:
    \[
    d_{\text{average}}(C_i, C_j) = \frac{1}{|C_i| \cdot |C_j|} \sum_{x \in C_i} \sum_{y \in C_j} \text{dist}(x, y)
    \]
    \item \textit{质心法}(Centroid Method):两个簇之间的距离定义为它们质心之间的距离:
    \[
    d_{\text{centroid}}(C_i, C_j) = \text{dist}(\mu_i, \mu_j)
    \]
\end{itemize}

\textbf{Ward 法}:Ward 法通过最小化每次合并后簇内方差的增加来决定簇的合并顺序:
\[
d_{\text{ward}}(C_i, C_j) = \frac{|C_k|}{|C_k| + |C_j|} \|\mu_k - \mu\|^2 + \frac{|C_j|}{|C_k| + |C_j|} \|\mu_j - \mu\|^2
\]

层次聚类的结果通常以树状图的形式展示,树状图展示了数据点合并或分裂的过程。通过剪切树状图可以得到不同数量的簇。层次聚类的优点在于它不需要预先指定簇的数量,并且能够生成一个多层次的聚类结果。缺点在于算法的计算复杂度较高,特别是在处理大规模数据集时。

HC 算法的时间复杂度通常为 \(O(n^2 \log n)\),其中 \(n\) 为样本数量。在某些情况下,复杂度可以达到 \(O(n^3)\)。

\subsection{OPTICS 聚类算法}

OPTICS(Ordering Points To Identify the Clustering Structure)是一种基于密度的聚类算法,用于识别任意形状和密度变化的簇。OPTICS 算法通过计算每个数据点的可达距离(Reachability Distance)和核心距离(Core Distance),并按照可达距离对数据点排序,从而构建簇的层次结构。首先,算法计算每个数据点的核心距离,即它到其邻域内满足最小点数要求的最远距离。然后,计算数据点之间的可达距离,即到达某个点所需的最小核心距离。根据可达距离对数据点进行排序,生成簇的层次结构。

核心距离的计算公式为:
\[
\text{core\_dist}(p) = \min_{p' \in \text{Neighbors}(p)} \text{dist}(p, p')
\]
可达距离的计算公式为:
\[
\text{reachability\_dist}(o, p) = \max(\text{core\_dist}(p), \text{dist}(o, p))
\]

其中,\(\text{dist}(p, p')\) 是数据点 \(p\) 和 \(p'\) 之间的距离,\(\text{Neighbors}(p)\) 是点 \(p\) 的邻域。

OPTICS 算法的优点在于它能够处理任意形状和密度的簇,并且无需预先指定簇的数量。缺点是算法在处理大规模数据集时计算复杂度较高,特别是在高维空间中。OPTICS 算法的时间复杂度为 \(O(n \log n)\),其中 \(n\) 为样本数量。虽然与 DBSCAN 类似,但 OPTICS 能更好地处理具有密度变化的复杂数据集。

\subsection{BIRCH 聚类算法}

BIRCH(Balanced Iterative Reducing and Clustering using Hierarchies)是一种适用于大规模数据集的聚类算法。BIRCH 算法通过构建和维护一个簇特征树(Clustering Feature Tree, CF Tree)来有效地对大规模数据进行聚类。CF Tree 是一种高度压缩的树状数据结构,能够通过增量式学习动态地调整簇。

BIRCH 算法的核心在于簇特征(Clustering Feature, CF)的计算和使用。簇特征 CF 是一个三元组 \((N, \vec{LS}, SS)\),用于有效地描述簇中的数据点。具体计算如下:

\[
\text{CF} = (N, \vec{LS}, SS)
\]
其中,\(N\) 是簇中的点数,\(\vec{LS}\) 是簇内所有数据点的线性和:
\[
\vec{LS} = \sum_{i=1}^{N} \vec{x}_i
\]
\(SS\) 是簇内所有数据点的平方和:
\[
SS = \sum_{i=1}^{N} \|\vec{x}_i\|^2
\]

通过簇特征,簇的质心和半径可以计算为:
\[
\vec{C} = \frac{\vec{LS}}{N}, \quad \text{Radius} = \sqrt{\frac{SS}{N} - \left(\frac{\vec{LS}}{N}\right)^2}
\]

两个簇 \(CF_1\) 和 \(CF_2\) 之间的距离可以通过以下公式计算:
\[
D(CF_1, CF_2) = \sqrt{\frac{N_1 \cdot N_2}{N_1 + N_2}} \cdot \|\vec{C}_1 - \vec{C}_2\|
\]

BIRCH 算法的主要步骤如下:
\begin{itemize}
    \item 构建 CF Tree:通过遍历数据集,BIRCH 算法将每个数据点插入到 CF Tree 中,并根据簇特征进行调整。
    \item 聚类阶段:在 CF Tree 构建完成后,BIRCH 可以使用凝聚层次聚类或其他算法对叶节点进行进一步的聚类,以生成最终的簇。
\end{itemize}

BIRCH 的优点在于它能够有效地处理大规模数据,并且能够在内存受限的情况下进行聚类。缺点是对于非球形簇的识别能力有限,并且可能对簇的初始构建顺序较为敏感。

BIRCH 算法的时间复杂度通常为 \(O(n)\),其中 \(n\) 为样本数量,适合大规模数据集的聚类任务。

\end{document}

