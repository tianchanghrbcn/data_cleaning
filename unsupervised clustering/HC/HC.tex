\documentclass{ctexart} % 使用ctexart文档类,自动处理中文

\usepackage{amsmath} % 支持数学公式

\begin{document}

\subsection{Hierarchical Clustering (HC) 聚类算法}

层次聚类(Hierarchical Clustering, HC)是一种用于分析数据集内嵌层次结构的聚类算法。HC 通过反复地将数据点进行分割或合并来形成一个树状的簇结构,称为树状图(Dendrogram)。HC 的两种主要方法是自底向上(凝聚法)和自顶向下(分裂法)。

\textbf{凝聚法}(Agglomerative Method):从每个数据点自身作为一个簇开始,逐步合并最近的簇,直到所有点都被合并到一个簇中。合并步骤通常基于以下几种距离度量:

\begin{itemize}
    \item \textit{最小距离法}(Single Linkage):两个簇之间的距离定义为它们之间最近点的距离:
    \[
    d_{\text{single}}(C_i, C_j) = \min_{x \in C_i, y \in C_j} \text{dist}(x, y)
    \]
    \item \textit{最大距离法}(Complete Linkage):两个簇之间的距离定义为它们之间最远点的距离:
    \[
    d_{\text{complete}}(C_i, C_j) = \max_{x \in C_i, y \in C_j} \text{dist}(x, y)
    \]
    \item \textit{平均距离法}(Average Linkage):两个簇之间的距离定义为它们之间所有点的平均距离:
    \[
    d_{\text{average}}(C_i, C_j) = \frac{1}{|C_i| \cdot |C_j|} \sum_{x \in C_i} \sum_{y \in C_j} \text{dist}(x, y)
    \]
    \item \textit{质心法}(Centroid Method):两个簇之间的距离定义为它们质心之间的距离:
    \[
    d_{\text{centroid}}(C_i, C_j) = \text{dist}(\mu_i, \mu_j)
    \]
\end{itemize}

\textbf{Ward 法}:Ward 法通过最小化每次合并后簇内方差的增加来决定簇的合并顺序:
\[
d_{\text{ward}}(C_i, C_j) = \frac{|C_k|}{|C_k| + |C_j|} \|\mu_k - \mu\|^2 + \frac{|C_j|}{|C_k| + |C_j|} \|\mu_j - \mu\|^2
\]

层次聚类的结果通常以树状图的形式展示,树状图展示了数据点合并或分裂的过程。通过剪切树状图可以得到不同数量的簇。层次聚类的优点在于它不需要预先指定簇的数量,并且能够生成一个多层次的聚类结果。缺点在于算法的计算复杂度较高,特别是在处理大规模数据集时。

HC 算法的时间复杂度通常为 \(O(n^2 \log n)\),其中 \(n\) 为样本数量。在某些情况下,复杂度可以达到 \(O(n^3)\)。


\end{document}
