\documentclass{ctexart} % 使用ctexart文档类,自动处理中文

\usepackage{amsmath} % 支持数学公式

\begin{document}

\subsection{BIRCH 聚类算法}

BIRCH(Balanced Iterative Reducing and Clustering using Hierarchies)是一种适用于大规模数据集的聚类算法。BIRCH 算法通过构建和维护一个簇特征树(Clustering Feature Tree, CF Tree)来有效地对大规模数据进行聚类。CF Tree 是一种高度压缩的树状数据结构,能够通过增量式学习动态地调整簇。

BIRCH 算法的核心在于簇特征(Clustering Feature, CF)的计算和使用。簇特征 CF 是一个三元组 \((N, \vec{LS}, SS)\),用于有效地描述簇中的数据点。具体计算如下:

\[
\text{CF} = (N, \vec{LS}, SS)
\]
其中,\(N\) 是簇中的点数,\(\vec{LS}\) 是簇内所有数据点的线性和:
\[
\vec{LS} = \sum_{i=1}^{N} \vec{x}_i
\]
\(SS\) 是簇内所有数据点的平方和:
\[
SS = \sum_{i=1}^{N} \|\vec{x}_i\|^2
\]

通过簇特征,簇的质心和半径可以计算为:
\[
\vec{C} = \frac{\vec{LS}}{N}, \quad \text{Radius} = \sqrt{\frac{SS}{N} - \left(\frac{\vec{LS}}{N}\right)^2}
\]

两个簇 \(CF_1\) 和 \(CF_2\) 之间的距离可以通过以下公式计算:
\[
D(CF_1, CF_2) = \sqrt{\frac{N_1 \cdot N_2}{N_1 + N_2}} \cdot \|\vec{C}_1 - \vec{C}_2\|
\]

BIRCH 算法的主要步骤如下:
\begin{itemize}
    \item 构建 CF Tree:通过遍历数据集,BIRCH 算法将每个数据点插入到 CF Tree 中,并根据簇特征进行调整。
    \item 聚类阶段:在 CF Tree 构建完成后,BIRCH 可以使用凝聚层次聚类或其他算法对叶节点进行进一步的聚类,以生成最终的簇。
\end{itemize}

BIRCH 的优点在于它能够有效地处理大规模数据,并且能够在内存受限的情况下进行聚类。缺点是对于非球形簇的识别能力有限,并且可能对簇的初始构建顺序较为敏感。

BIRCH 算法的时间复杂度通常为 \(O(n)\),其中 \(n\) 为样本数量,适合大规模数据集的聚类任务。


\end{document}
