\documentclass{ctexart} % 直接使用ctexart文档类,自动处理中文
\begin{document}

\subsection{Affinity Propagation (AP) 聚类算法}

Affinity Propagation (AP) 是一种基于消息传递的聚类算法,通过在数据点之间传递“责任”(Responsibility, \(r(i, k)\))和“可用性”(Availability, \(a(i, k)\))消息,自动确定簇的数量和中心点。算法首先计算数据点之间的相似度 \(s(i, k)\),然后初始化责任和可用性矩阵。

AP 的目标是通过不断更新以下公式,使得责任和可用性值达到平衡:

\[
r(i, k) = s(i, k) - \max_{k' \neq k} \{a(i, k') + s(i, k')\}
\]
\[
a(i, k) = \min\left(0, r(k, k) + \sum_{i' \notin \{i, k\}} \max(0, r(i', k))\right)
\]

其中,责任 \(r(i, k)\) 表示数据点 \(i\) 作为簇中心 \(k\) 的适合度,可用性 \(a(i, k)\) 表示数据点 \(i\) 选择 \(k\) 作为簇中心的适合度。最终具有较高“可用性”和“责任值”的点被选择为簇中心。

AP 的优点在于不需要预先指定簇的数量,且能够处理任意形状的簇。缺点在于对相似度度量和偏好值敏感,且计算复杂度较高。AP 算法的时间复杂度为 \(O(n^2 \cdot \log(n))\),其中 \(n\) 为样本数量。
\end{document}