\documentclass{article}
\usepackage{amsmath}  % 包含数学公式的包
\usepackage{xeCJK}    % 处理中文的包,适用于XeLaTeX
\setCJKmainfont{SimSun} % 设置中文字体,如宋体

\begin{document}

\subsection{K-Means 聚类算法}

K-Means 是一种基于距离的聚类算法,通过迭代将数据集划分为 \(K\) 个簇。算法首先随机选择 \(K\) 个初始质心,然后将每个数据点分配到最近的质心,接着更新质心为簇内数据点的均值。这个过程重复进行,直到质心位置稳定或达到最大迭代次数为止。

K-Means 的目标是最小化簇内数据点到其质心的总距离平方和,具体表达为:

\[
J = \sum_{k=1}^{K} \sum_{i \in C_k} \| x_i - \mu_k \|^2
\]

其中,\(x_i\) 表示数据点,\(\mu_k\) 为第 \(k\) 个簇的质心,\(C_k\) 是簇 \(k\) 中的所有数据点。通过最小化 \(J\) 值,算法不断优化簇的划分。

K-Means 算法的优点在于简单易实现,且在处理大规模数据集时效率较高。然而,它需要预先指定簇的数量 \(K\),且对初始质心选择较为敏感,不同的初始选择可能导致不同的聚类结果。此外,K-Means 更适用于凸形簇,对噪声和异常值的处理能力较弱。

K-Means 的平均时间复杂度为 \(O(K \cdot n \cdot T)\),其中 \(K\) 为簇数,\(n\) 为样本数量,\(T\) 为迭代次数。最坏情况下,复杂度为 \(O(n^{(K+2)/p})\),其中 \(p\) 为特征数量。处理高维数据时,算法的计算开销可能会增加,尤其是在初始质心选择不当的情况下。

\end{document}
